\documentclass{minimal}
\begin{document}
Wire gauge is the resistance per unit length in decibels, where 10 AWG
wire has 1 ohm per thousand feet resistance.  Therefore, every
increase of 3 AWG ($\approx 10\times\log_{10}2$) increases the resistance
per unit length by a factor of two.  The general formula for the gauge
of wire given the resistance per thousand feet $R$ is
\begin{displaymath}
\mathrm{AWG} = 10 \times\log_{10}R + 10
\end{displaymath}
or
\begin{displaymath}
R = 10^{\mathrm{AWG}-10 \over 10}
\end{displaymath}
ohms per thousand feet.

Wire of length $L$, cross-sectional area $A$ and conductivity $\sigma$
has resistance $L/\sigma A$, or resisitance per unit length $R =
1/\sigma A$.  Using the formula above gives
\begin{displaymath}
A \propto 10^{-{\mathrm{AWG}-10 \over 10}}\qquad\mathrm{or}\qquad 
  D \propto 10^{-{\mathrm{AWG}-10 \over 20}}
\end{displaymath}
where $D$ is the diameter of the wire.  Using the fact that 10 AWG
wire has a diameter of 0.10 inches gives
\begin{displaymath}
D = 10^{-{\mathrm{AWG}+10 \over 20}}.
\end{displaymath}
Solving this equation for wire gauge given diameter
\begin{displaymath}
\mathrm{AWG}=-20\times\log_{10} D - 10.
\end{displaymath}

A reasonable value for the current carrying capacity of a wire can be
estimated from the rule that the cross-sectional area should be 700
``circular mils'' per amp.  The (rather perverse) definition of a
circular area measured in ``circular mils'' is that it is the area of
the square whose side has the same length as the diameter of the
circle, i.e. just the diameter (in mils or 0.001 inch) squared.  The
current carrying capacity in amps is therefore (with $D$ measured in
inches)
\begin{displaymath}
I={10^{6}\times D^2\over 700}
\end{displaymath}
or in terms of AWG
\begin{displaymath}
I={1\over 700}\times10^{50-\mathrm{AWG}\over 10}
\end{displaymath}
solving for AWG in terms of current
\begin{displaymath}
\mathrm{AWG}=50 - 10\times\log_{10}(700\times I)
\end{displaymath}

The power dissipated in the wire per foot is
\begin{displaymath}
P=I^2R.
\end{displaymath}
Substituting $I$ and $R$ given by the formulas above gives
\begin{displaymath}
P=\Big({1\over 700}\Big)^2\times10^{90-\mathrm{AWG}\over 10}
\end{displaymath}
Watts per thousand feet.
\end{document}
